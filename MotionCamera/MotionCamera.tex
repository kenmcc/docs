\documentclass[a4paper]{article}
\usepackage[a4paper, top=2cm, bottom=2cm, left=2cm, right=2cm]{geometry}
\usepackage{pdfpages}
\begin{document}
\tableofcontents
\newpage
\section {Introduction to Python}
Python is a high-level programing language with a lot of powerful capabilities. 
It is simple to learn and fast to develop in, and is ideally suited for many tasks.
Here in S3 we use it for both commercial deployment, as well as internal tasks such as document generation, test script execution and personal projects.
\\
\\
There are plenty of tutorials on the internet for reference, e.g. \\
http://sthurlow.com/python/

Python looks very like English and therefore is very simple to write. 


\newpage\section {Interacting with real world items}
Many computer programs do not really ``do'' much - they run on a PC and never really interact with the real world. 
We are going to develop a simple motion-detecting device which will.

\begin{itemize}
\item Wait for some movement
\item Take a photograph
\item Email it to an email address (this bit is a bit limited here in S3, it can't send an email outside the company)
\end{itemize}

We are going to do so using the Raspberry Pi, a cheap, very powerful micro computer which 
has been designed especially to make computing accessible to students and people wishing 
to learn about coding and software.
\\
\\
In order to detect the motion, we will use a P.I.R - a Passive Infrared Receiver. 
These are in use in all sorts of things like burgler alarm sensors, security lights, 
even the glade room-scent plug-ins which release an aroma as you walk past. They are very cheap, only a few euro on Ebay or similar.
\\
\\
To take the picture we will use the Raspberry Pi Camera add-on to take the photo.
\\
\\
Using the power of Python, we shall be able to send am email to one or more addresses, with the photo attached. 
So you could set this up in your room and know if someone has been into it while you are away.
\\
\\
Of course there are other things that a setup like this could do - we could change the behaviour of this so that 
instead of taking a picture and emailing it, it could be used to turn on a light when you enter a room instead. 
The possibilities are endless.
\\
\\
Lets go....


\newpage\section {Detecting Movement}
The first stage is to get the raspberry pi to detect movement. A PIR sensor is connected to the board with 3 wires.
\begin{itemize}
\item+5v
\item 0V
\item Data
\end{itemize}
+5v and 0 power the PIR sensor, and the Data connection is used to tell the Raspberry Pi when the sensor has detected movement
\\
\\
Reference : http://www.raspberrypi-spy.co.uk/2013/01/cheap-pir-sensors-and-the-raspberry-pi-part-1/
\\
\\
The code from this example is on the Raspberry Pi already, in the file ``pir\_sensor.py''
\\
\\
This code does some small amount of setup, and then enters a loop which continuously reads the value of the data pin from the sensor.\\
If this value is ``1'', then the sensor is indicating that it has detected some movement. \\
If the values is ``0'', then the sensor has not detected any movement \\
\\
\\
We can run this code by typing \\
\emph{\textbf{sudo python pir\_sensor.py}}
\\
So this script will sit there and print ``Motion Detected!'' when it notices movement and ``Ready'' when it has reset and is waiting for the next movement to occur
\\
\\
To exit the program, type the Ctrl Button and 'C'
\\
\\
\subsection{Tasks:}
\begin{enumerate}
\item Run this script, as indicated above, and ensure that when you move near the sensor it detects it, and does not do so when you are still.
\item Edit the file and change the messages which are printed out when motion is detected, and when it is ready to detect again.
\end {enumerate}

\newpage\section {Taking a photo}
Now that we can detect motion, it's time to take a photo. In order to do this we use a python building block, called picamera, to take the photo.
\\
\\
Reference: https://www.raspberrypi.org/picamera-pure-python-interface-for-camera-module/
\\
\\
The code for this part is on the Raspberry Pi already, in the file ``takephoto.py''
\\
\\
We can run this by typing \\
\emph{\textbf{python takephoto.py}}
\\
And it should create a picture called ``image.jpg'' which contains an image of the area the camera is pointing at.
\\
\\
It's a very simple piece of code, all it does is say \\
- use the picamera module \\
- take a picture and call it ``image.jpg'' \\
- say when you've done that
\\
\\
\subsection{Tasks:}
\begin{enumerate}
\item Run this script, as indicated above, and ensure that it takes a photo
\item Edit the file and change the name of the photo it creates from ``image.jpg'' to ``motion.jpg''
\end {enumerate}

\newpage\section{Sending an Email}
Python includes a lot of powerful functions to do all sorts of wonderful things very simply, and one of these is to send an email
\\
Reference : https://docs.python.org/2/library/email-examples.html
\\
\\
There's a script on the raspberry pi which sends an email with an attachment
\\
We can run this with \\
\emph{\textbf{python emailer.py}} \\
but this will fail as we have to tell it who to send the email to \\
\\
So edit this file and change the line that says \\
to=None\#``your.name@s3group.com'' \\
and make it\\
to=``yourEmailAddress@s3group.com''\\
e.g. \\
to=``ann.smith@mail.ie''\\
\\
and save this. Now when we run it \\
\emph{\textbf{python emailer.py}} \\
it will try to send the photo to your email address. But we changed the name of the photo from ``image.jpg'' to ``motion.jpg'', 
so we need to change the emailer script again so that it uses ``motion.jpg''. \\ 

\subsection{Tasks:}
\begin{enumerate}
\item Run the script and verify that it sends an email with the photo
\item Change the script so that the subject line reads ``INTRUSION ALERT'' and that the email body says ``Motion has been detected near the observation zone'', instead of ``See for yourself''
\item Take a few different photos using the ``takephoto.py'' script and email them using the ``emailer.py'' script
\end{enumerate}

\newpage\section {Tying it all together}
So now we have all the individual pieces of the puzzle which allow us to:
\begin{enumerate}
\item Detect Motion
\item Take a photo
\item Email the evidence
\end{enumerate}

It's time to tie it all together into one program which does it all. 
\subsection{Task:}
\begin{enumerate}
\item Create a new file called ``motioncamera.py''
\item Using the code from the 3 other files, create a program which waits for motion to be detected, and when it does so, take a picture and email it to you
\end{enumerate}    
Some issues you may encounter : \\ 
\\
Spaces are very important in python so you may get an error saying ``unexpected indent'' or similar\\ 
\\
You might get an error like ``picamera.exc.PiCameraMMALError: Camera component couldn't be enabled: Out of resources (other than memory)'' when trying to take a photo. 
This means that you have already opened the camera (camera = picamera.PiCamera()) - there is only one camera so it only needs to be opened once. \\
\\
You might see an error like this : ``RuntimeError: No access to /dev/mem.  Try running as root!'' when trying to run the program. Use ``sudo python motioncamera.py'' instead \\
\\
\textbf{ANY QUESTIONS JUST ASK!}


\newpage\section{Other things to do}
If we get all this done, there are other things we could do - for example
\begin{enumerate}
\item Take a short video rather than a photo, and email that
\item Change the mechanism so that rather than a PIR, this is connected to a button so that it could be used as a doorbell
\end{enumerate}







\end{document}
